% THIS IS AN EXAMPLE DOCUMENT FOR VLDB 2012
% based on ACM SIGPROC-SP.TEX VERSION 2.7
% Modified by  Gerald Weber <gerald@cs.auckland.ac.nz>
% Removed the requirement to include *bbl file in here. (AhmetSacan, Sep2012)
% Fixed the equation on page 3 to prevent line overflow. (AhmetSacan, Sep2012)

\documentclass{sig-alternate}
\usepackage{graphicx}
\usepackage{color}
\usepackage{balance}  % for  \balance command ON LAST PAGE  (only there!)
\usepackage{times}
\usepackage{url}
\usepackage{algorithm}
\usepackage[noend]{algorithmic}
\usepackage{subfigure}
\usepackage{xspace}
\usepackage[noend]{algorithmic}
\usepackage{enumerate}
\usepackage{multirow}
\usepackage{epstopdf}
\usepackage{cleveref}
\newcommand{\squishlist}{
   \begin{list}{$\bullet$}
    {
      \setlength{\itemsep}{0pt}
      \setlength{\parsep}{3pt}
      \setlength{\topsep}{3pt}
      \setlength{\partopsep}{0pt}
      \setlength{\leftmargin}{1.5em}
      \setlength{\labelwidth}{1em}
      \setlength{\labelsep}{0.5em} } }

\newcommand{\squishend}{
    \end{list}  }

\newcommand{\argmax}{\operatornamewithlimits{arg\ max}}

\newcommand{\eat}[1]{}
\newcommand{\todo}[1]{\textcolor{red}{{TODO: #1}}}
\newcommand{\add}[1]{\textcolor{red}{{ADD: #1}}}
\newcommand{\note}[1]{\textcolor{blue}{{#1}}}

\newtheorem{definition}{Definition}
\newtheorem{example}{Example}
\newtheorem{theorem}{Theorem}
\newtheorem{lemma}{Lemma}
\newtheorem{problem}{Problem}
\newtheorem{reduction}{Reduction}
\newcommand{\domain}{\mathcal{D}}
\newcommand{\attributes}{\mathcal{A}_D}
\newcommand{\hierarchy}{\mathcal{H}_D}
\newcommand{\attrhierarchy}{\mathcal{H}_A}
\newcommand{\workers}{\mathcal{W}}
\newcommand{\uentities}{\mathcal{E}}
\newcommand{\queryvector}{{\bf Q_S}}



\begin{document}

% ****************** TITLE ****************************************

\title{CrowdGather: Budgeted Entity Extraction over Hierarchies}

\numberofauthors{3} 

\author{
}

\maketitle

\begin{abstract}
Crowd entity extraction has become a popular means of acquiring data for many applications, including recommendation systems, listing aggregation and knowledge base compilation.  Most of the current solutions focus on entity extraction over spa queries in isolation and do not consider entire entity domains. 

However, many of the current solutions due to the time and cost of human labor, current solutions may incur large costs, thus, limiting their applicability. In this paper, we explore the problem of 
\end{abstract}

\section{Introduction}
\subsection{Challenges}
\subsection{Contributions}

\section{Crowdsourced Entity Extraction}
\subsection{Budgeted Entity Extraction}
\subsection{Extracting Entities over Hierarchies}

\section{Framework Overview}
\subsection{}
\subsection{}
\subsection{}

\section{The Gain of Further Queries}

\section{Profitable Querying Policies}

\section{Conclusions}


\bibliographystyle{abbrv}
\bibliography{crowd_hierarchies}

\end{document}
